\documentclass[a4paper, 11pt]{scrartcl}                % final (journal style)
%\documentclass[review,journal]{vgtc}         % review (journal style)
%\documentclass[widereview]{vgtc}             % wide-spaced review
%\documentclass[preprint,journal]{vgtc}       % preprint (journal style)
%\documentclass[electronic,journal]{vgtc}     % electronic version, journal
\usepackage[utf8]{inputenc} 
\usepackage[T1]{fontenc}  
\usepackage[left=2.5cm, right=2.5cm, top=2.5cm, bottom=3cm, a4paper]{geometry}
\usepackage[pdftex]{graphics} 
\usepackage{mathptmx}
\usepackage{graphicx}
%\usepackage{times}



\title{\Large Distributed file (storage?) system based on P2P networks}
\author{\small Divya Avalur, Christian Manteuffel, and Spyros Ioakeimidis}
\date{\small \today}

\setlength{\parindent}{0pt} 

\begin{document}

\maketitle

\section{Introduction}

any-sized company needs a distributed file system

many local networks exist internally in the companies, wireless or not

these local networks could be used for the realization of the distributed file system

or a cloud distributed file system

however ...

how is this different from a local network shared directory?

\section{Project}

Distributed file storage system based on peer-to-peer networks, independently from the kind of network.

Peers share a local distributed file system.

Files are replicated across the peers of the network.

Every copy of the file must be the same.

When a peer joins the network, then automatically the files are synchronized.

Security stays at the level of the network security.

Applicable for companies, which are in the need of a local distributed file system.

\section{Challenges}

\begin{itemize}
	\item Replication
	\item Synchronization
	\item Concurrency
	\item Failure Handling
	\item Conflict Handling
	
	When a peer leaves network but keeps working in files, what happens when it rejoins the network?
\end{itemize}

\end{document}