\documentclass[a4paper, 11pt]{scrartcl}                % final (journal style)
%\documentclass[review,journal]{vgtc}         % review (journal style)
%\documentclass[widereview]{vgtc}             % wide-spaced review
%\documentclass[preprint,journal]{vgtc}       % preprint (journal style)
%\documentclass[electronic,journal]{vgtc}     % electronic version, journal
\usepackage[utf8]{inputenc} 
\usepackage[T1]{fontenc}  
\usepackage[left=2.5cm, right=2.5cm, top=2.5cm, bottom=3cm, a4paper]{geometry}
\usepackage[pdftex]{graphics} 
\usepackage{mathptmx}
\usepackage{graphicx}
%\usepackage{times}



\title{\Large Ad-hoc peer-to-peer shared file storage}
\author{\small Divya Avalur, Christian Manteuffel, and Spyros Ioakeimidis}
\date{\small \today}

\setlength{\parindent}{0pt} 

\begin{document}

\maketitle

\section{Project Proposal}

In general, shared filesystems rely on a server-client scheme, where the actual files are stored on one server. 
The disadvantage is that the files reside on the server and are not automatically distributed to the clients. This has to be done explicitly by the user, whether he wants to add new files or work on existing files.
In order to share files, the computers must be connected to the server.
In our project, we imagine a distributed filesystem based on a peer-to-peer network that operates without a central server. 
The actual files do not reside on one particular computer, but instead are spread across multiple computers. All computers together form a virtual filesystem.

The goal of the project is to create an ad-hoc peer-to-peer network that synchronizes files between  computers. In this case ad-hoc means that as soon as a computer joins a network, it should search for other peers and start to exchange files. Eventually all computers have the same copy of all files. 
The system should be designed for situations, in which computers reside only a short time in the same network. For example, a group of students that meets once a day. During that meeting, the system should automatically synchronize the files to the latest version. 
Hence, the system needs to keep track of file changes and needs to be able to determine the latest version of a file. 

\section{Problems and challenges}

\begin{itemize}
    \setlength{\itemsep}{0pt} \setlength{\parskip}{0pt}

	\item Automatically finding other peers within an arbitrary network.
	\item Determining the latest version of a file and avoid conflicts.  For example, what happens when two peers changed the same file?
	\item Creating a reliable and fault-tolerant middleware.
	\item Consider security and privacy concerns. e.g.\ a shared key as access token. 
	\item Synchronize files across different network topologies, e.g.\ not all peers are in the same network.
	 \item Is the system applicable to a larger network, e.g.  the Internet? What needs to be changed? 
\end{itemize}

\section{Literature / Similar Projects}

\begin{itemize}
	\item Hasan, R.; Anwar, Z.; Yurcik, W.; Brumbaugh, L.; Campbell, R.; , "A survey of peer-to-peer storage techniques for distributed file systems," Information Technology: Coding and Computing, 2005. ITCC 2005. International Conference on , vol.2, no., pp. 205- 213 Vol. 2, 4-6 April 2005
doi: 10.1109/ITCC.2005.42
	\item Peric, D.; Bocek, T.; Hecht, F.V.; Hausheer, D.; Stiller, B.; , "The Design and Evaluation of a Distributed Reliable File System," Parallel and Distributed Computing, Applications and Technologies, 2009 International Conference on , vol., no., pp.348-353, 8-11 Dec. 2009
doi: 10.1109/PDCAT.2009.37
	\item http://pdos.csail.mit.edu/ivy/; https://www.aerofs.com/features; http://code.google.com/p/perfs/
	
\end{itemize}

\end{document}