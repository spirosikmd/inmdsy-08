%!TEX root = ../Peerbox.tex

Peerbox is designed to be used for small groups as a simple distributed file service.

% Access transparency: good - access control per directory, not per
%file
%Access transparency: excellent - the API is the UNIX system call
%interface for both local and remote files.
%Location transparency: not guaranteed but normally achieved -
%naming of filesystems is controlled by client mount operations, but
%good location transparency can be ensured by an appropriate
%system configuration.
%
%Mobility transparency: hardly achieved - relocation of files is not
%possible, relocation of filesystem is possible, but requires updates
%to client configurations.
% Mobility transparency: good

% Location transparency: poor, but not noticeable by users, only by
%system administrators

One of the most important requirements is the kind and level of transparency that Peerbox provides to the external environment. Peerbox offers \textbf{weak access transparency}. The users can use the OS specific file service in order to interact with the local files. However, in order to interact with remote files they would need to use the Peerbox interface. In order to achieve this goal, we would need to integrate Peerbox with the OS file service.  Access control is provided by the operating system and Peerbox does not introduce additional access control at the moment.

Peerbox offers \textbf{weak location transparency}. The fact that the peers are aware that some files of the Virtual File System are not stored locally, and that to which members these files belong to, it does decrease the level of location transparency. The peers are aware of file distribution.





%Replication: limited, only on read-only files

% Replication: limited to read-only file systems. File replication for
%updates is achieved with a support of the Sun Network
%Information Service (NIS).



% Security: limited, Kerberos authentication protocol used
% 
%  Security: hardly achieved - no data is encrypted and UNIX-style
%authentication is used. Additionally, it supports Diffie-Hellman
%and Kerberos authentication mechanisms.




As Peerbox is a peer-to-peer application, the \textbf{performance is good} but it is dependent on the levels offered by each peer and by the ad-hoc network. If the peers contain multiprocessor machines then the performance will be high. Ad-hoc networks are usually very fast.



% Scaling: good - a group of NFS servers can handle very large loads.
%For a single server, the performance limit is determined by its load.
%Additional servers may be added, so that filesystems are
%subdivided and allocated between them.
 %Scalability: excellent

Peerbox exhibits \textbf{good scalability}. It can scale linearly and for every new peer the number of messages is increased by $n*m$, where $n$ is the number of peers and $m$ is the number of messages. 



Peerbox provides \textbf{excellent heterogeneity} as it can be accessed by clients running on MacOS, Windows, Linux and any hardware platform as long as the Java dependency is met. Furthermore, the design of Perebox is compatible with the file systems of different OSes.

% Failure: limited but effective - file service is suspended if a server
%fails. Failure recovery is aided by the simple stateless design.




The level of provided \textbf{concurrency is limited} but adequate for most cases. All the peers share the same state of the file system due to the use of the Virtual File System in the application logic. Peerbox offers file-level locking which is a sufficient locking mechanism for the purposes that it is meant to be used. It is likely that when read-writes are shared concurrently consistency would not be perfect.

Security is not addressed by Peerbox and it is left to the levels of the operating system.


