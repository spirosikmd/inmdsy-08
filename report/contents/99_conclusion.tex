%!TEX root = ../Peerbox.tex

Peerbox designed to be used for small groups as a simple and robust distributed file service. As such it needs to address some specific requirements.\\

One of the most important requirements is the kind and level of transparency that it provides to the external environment. Peerbox offers good Access Transparency. On the one hand, the API is not operating system dependent, and it works transparently for both local and remote files. On the other hand, the fact that the peers are aware that some files of the Virtual File System are not stored locally, it does decreases the level of transparency. Access control is provided by the operating system and Peerbox does not introduce additional access control at the moment.



% Access transparency: good - access control per directory, not per
%file
%Access transparency: excellent - the API is the UNIX system call
%interface for both local and remote files.
%Location transparency: not guaranteed but normally achieved -
%naming of filesystems is controlled by client mount operations, but
%good location transparency can be ensured by an appropriate
%system configuration.
%
%Mobility transparency: hardly achieved - relocation of files is not
%possible, relocation of filesystem is possible, but requires updates
%to client configurations.
% Mobility transparency: good

% Location transparency: poor, but not noticeable by users, only by
%system administrators




 %Scalability: excellent
 
 


%Replication: limited, only on read-only files

% Replication: limited to read-only file systems. File replication for
%updates is achieved with a support of the Sun Network
%Information Service (NIS).



% Security: limited, Kerberos authentication protocol used
% 
%  Security: hardly achieved - no data is encrypted and UNIX-style
%authentication is used. Additionally, it supports Diffie-Hellman
%and Kerberos authentication mechanisms.





%Performance: excellent - dramatically reduced loads on servers. A
%server load of 40\% with 18 client nodes versus a load of 100\% for
%NFS
%
% Performance: good - multiprocessor servers achieve very high
%performance, but for a single filesystem it is not possible to go
%beyond the throughput of a multiprocessor server.





% Scaling: good - a group of NFS servers can handle very large loads.
%For a single server, the performance limit is determined by its load.
%Additional servers may be added, so that filesystems are
%subdivided and allocated between them.



% Heterogeneity: good - NFS is implemented for almost every OS
%and hardware platform.

% Failure: limited but effective - file service is suspended if a server
%fails. Failure recovery is aided by the simple stateless design.

% Concurrency: limited but adequate for most purposes - when
%read-write files are shared concurrently between clients,
%consistency is not perfect.



