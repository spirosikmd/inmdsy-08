%!TEX root = ../Peerbox.tex

\subsection{Idea}
%manni

Christian, Divya, and Spyros are three students who work on a project in Security Systems. The ideal for them is to work together. This means that they need to share their work with each other. However, access to the internet is not always given or the internet does not work reliably. Nevertheless, the students need to work effectively by sharing their work --- this means that they should be able to access files that are stored on another computer.

computers are combined in a virtual fs that automatically adds files.

this vfs uses a peer-2-peer scheme in order to automatically add peers to the vfs. 

The proposed solution should be able to work across network boundaries, creating a flexible structure and thus the peer do not have to be physically connected to each other. 

this way, every student does not need to transmit ownership of a file because the file is stored locally on his computer and filesharing is possible with minimal configuration. 



Figures showing three computers that build a vfs.


\subsection{Context}
%manni

\subsection{State of the Art}
%spyros

During the past decade, various peer-to-peer file systems were introduced. Most of them share common properties like availability, sharing, and safety. However, some more advanced peer-to-peer file systems offer synchronisation, and conflict detection and resolution. This section describes some of the existing peer-to-peer file systems.

\begin{description}
	\item[Pastis]\-\\
	Pastis \cite{Busca:2005gt} is a decentralised multi-user read-write peer-to-peer file system. Every file is described by
a modifiable inode-like structure which contains the addresses of the immutable blocks in which the file contents are stored.

All data are stored using the Past \gls{acr:dht}, which has been modified in order to reduce the number of network messages it generates, thus optimising replica retrieval. Pastis is known for its simplicity, high scalability, fault tolerance, and locality properties of its underlying storage layer.
	\item[Ivy]\-\\
	Ivy \cite{Muthitacharoen:2002iv} is a multi-user read and write peer-to-peer file system. It has no centralised or dedicated components. It provides useful integrity properties without requiring users to fully trust either the underlying peer-to-peer storage or other users of the file system.
	
	One of its properties is that with a special arrangement between logs and the modifications of participants, it maintains meta-data consistency without locking. Ivy provides semantics like \gls{acr:nfs}, and is able of detecting conflicting modifications. Performance measurements show that Ivy is two to three time slower than \gls{acr:nfs}.
	\item[ColonyFs]\-\\
	ColonyFS \cite{Colony:2009fs} is a distributed file system which emphasises anonymity, security and dependability over a peer to peer network. This system implements a technique called \gls{acr:frs}, and uses an optimisation algorithm inspired by the movement of ants. The aim of the project is to produce an implementation of these techniques for the specific requirements of a dynamic peer-to-peer network where participants can join and leave at will.
	\item[Infinit]\-\\
	Infinit\footnote{http://en.wikipedia.org/wiki/Infinit} is a peer-to-peer file system which allows users to store, access, and share files in a safe and collaborative way. It also allows the virtualisation of decentralised storage space as one coherent drive. Infinit ensures reliability by dynamically replicating the data so that devices can crash without incurring data loss. It also ensures privacy by access control mechanisms.
	
	Furthermore, it is known for ensuring synchronisation by distributing files and directories throughout the network's infrastructure and update them in real time. Some additional properties of Infinit are sharing, transparency, security, anonymity, and availability.
	\item[Darknet]\-\\
	Darknet \cite{Ledung:2010wq} is private peer-to-peer file system which aims to provide safe, fast, and scalable file sharing without constraining the users in this aspect. It utilises a decentralised peer-to-peer network overlay by creating a prototype with extreme programming as methodology. To maximise the freedom of users the network is accessed through a virtual file-system interface.
\end{description}
