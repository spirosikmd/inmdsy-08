%!TEX root = ../Peerbox.tex

\subsection{Example Scenarios}


% \subsubsection{Joining a peer to peer network}
% This scenario describes how a user joins a peer-to-peer network. The basic requirements includes a desktop PC or laptop and internet connection preferably Wi-fi. All the peers send messages via multicasting. For joining the network, one peer creates a new network connection and sends the password to all the remaining peers. All the peers accept the password and join the      network. 
\subsubsection{Modifying and updating files}
This scenario describes how the user modifies and updates the files in a P2P network. The peer can view the list of files created by other peers. In order to modify any specific file, the peer sends a file request to the other owning peer. The peer who has the requested file, sends the file immediately to the requested peer. The requested user then makes the respective changes and then updates it with a new version number.  

\subsubsection{Disjoining the peer-to-peer network}
This scenario describes how the user can leave the peer-to-peer network. If the user wishes to leave the network, he sends the leave message to all the peers. All the peers accept his request. In this way, a peer can safely leave the peer-to-peer network.

\subsection{Properties of Peerbox}
%spyros

While designing Peerbox, we needed to identify significant properties regarding a file system. For example, the size distribution of files, and the relative and absolute frequencies of different file operations. In order to identify these properties, we needed to imagine the context that Peerbox would be placed in and how Peerbox would be used.

\subsubsection{File distribution size}

There is no restriction on the size distribution of files. This literally means that even huge files could be shared, for instance large image, pdf or small video files. However, Peerbox is optimized for file sizes up to 100 MB \todo{not shore about this}. This can cover almost every kind of file type sharing. However, we do not expect that Peerbox will be used for sharing large image and video files.

\subsubsection{File operations}

Peerbox is intended to be used by groups of average size 3-5 members. It is expected that a group will work together for an average time of 2 hours. Usually during a group meeting, the members exchange files in order to evaluate the progress and discuss about it. Therefore, it is expected that file reads will be more than writes as each group member will need to read files from the rest of the members. This also results in a high file sharing rate.

The number of file creations is expected to be significantly higher than deletions. Usually in group meetings, new files are created in order to introduce new content. Files are rarely deleted until probably the end of a project. The number of file modifications is expected to be low, as usually during group meetings new content is not added.

\subsubsection{File type}

Peerbox does not introduce any restriction on the type of files that can be shared. This influences the property of the size distribution of files. Usually, the size of image and video files is large. This considerably increases the size distribution of files. However, It does not influence the property of file operations. The number of file read is not affected by the various file types.