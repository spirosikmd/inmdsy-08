%!TEX root = ../Peerbox.tex

\subsection{Example Scenarios}


% \subsubsection{Joining a peer to peer network}
% This scenario describes how a user joins a peer-to-peer network. The basic requirements includes a desktop PC or laptop and internet connection preferably Wi-fi. All the peers send messages via multicasting. For joining the network, one peer creates a new network connection and sends the password to all the remaining peers. All the peers accept the password and join the      network. 
\subsubsection{Modifying and updating files}
This scenario describes how the user modifies and updates the files in a P2P network. The peer can view the list of files created by other peers. In order to modify any specific file, the peer sends a file request to the other owning peer. The peer who has the requested file, sends the file immediately to the requested peer. The requested user then makes the respective changes and then updates it with a new version number.  

\subsubsection{Disjoining the peer-to-peer network}
This scenario describes how the user can leave the peer-to-peer network. If the user wishes to leave the network, he sends the leave message to all the peers. All the peers accept his request. In this way, a peer can safely leave the peer-to-peer network.

\subsection{Properties of Peerbox}
%spyros

While designing Peerbox, several decisions had to be made regarding some significant properties. For example, the size distribution of files, and the relative and absolute frequencies of different file operations.

\subsubsection{File distribution size}

The maximum size distribution of file that Peerbox can support is 25 MB. This will even allow the sharing of image and pdf files, which often have large sizes.

\todo{can or is optimized for? Why will it be limited to 25mb?}

\subsubsection{File operations}

During Peerbox operation it is excepted that the number of file modifications will be higher than the number related  to file reads. However, it is expected that there will also be a significant number of reads, as every peer of the group will need to read files from other peers. This means that sharing of files between peers is expected to be high. The number of file creations and deletions is expected to be relatively low.\\

\todo{you need some argumentation here, why are more modifications are expected than reads?}

Peerbox does not introduce any restriction on the type of files that can be shared. This influences the properties related to file operations. File modifications are expected to be slightly less than the previous expectation, mainly due to the existence of binary files that cannot very often modified. File creations are expected to increase, as the absence of restriction on the file type means that more files of various types can be created.